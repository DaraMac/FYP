\chapter{Conclusion}

\section{Contribution to the State-of-the-Art}

Other than in \cite{firing} there was not much evidence to be found of a project of this sort being attempted before.
This project certainly has similarities,
but trades the depth of \cite{firing} for breadth,
and makes use of definitions that aim more for compactness and reusability.
For instance the time parameter here is more external to the cellular automata,
and the varieties of different CA are each defined with different mechanisms on top of a shared common base.
In that respect it should be fair to say this project brings some novelty in its ideas,
or at least its application of those ideas to this specific problem domain.


\section{Results Discussion}

The models produced in this project are not all immediately generalisable. 
A large amount of tedious manual reworking would be necessary to extend to higher dimensions if the definitions were to follow their current patterns.

However the properties of these models that were proved on top of them show their robustness by applying easily to all these variations.
Also a lot of the results about rules could hold very generally with a few more abstractions made.

Some of the auxiliary functions necessary to get the two dimensional finite CA working are very specific to that use case and do not naturally extend to higher dimensions at all.

The results certainly do hold up when not looked at in terms of extending upwards in dimension,
but instead in terms of being able to build up even further inside their own dimension.
That is they are very suitable for continuing to develop new results on top of them.


\section{Future Work}

\begin{itemize}
    \item The CA could all be put under the umbrella of some unifying structure like a typeclass.
        This would reduce duplication in definitions and code,
        and greatly ease expansion of the project.
    \item The finite CA definitions could be rewritten in a way that would allow for easy extension to arbitrary finite dimensions.
        One possible avenue for this could be topologically based.
        The state could always be a single one dimensional list,
        and the neighbourhood function could be written to generate arbitrarily dimensioned topologies on top of it.
    \item An alternative approach for this could be taken from graph theory, 
        by making neighbours have an edge between them,
        you could get an arbitrary embedding of any neighbourhoods,
        in an efficient and simple graph data structure.
    \item It would be interesting to attempt formalisations of similar computational models but ones in a new family.
        Namely the abstract Tile Assembly Model (aTAM) \cite{doty} used for modelling DNA computing.
        It is relevant in that it is a geometric tiling model of computation,
        but it contains far more complications.
        This would make formalisation both harder and more worthwhile,
        as it can ensure definitions are correct and stamp out inconsistencies.
\end{itemize}
