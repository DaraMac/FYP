\chapter{The Solution}

\section{Architectural Level}

\begin{figure}[h]
    \centering
    \includegraphics[scale=0.62]{session_graph.pdf}
    \caption{Dependency graph of project Theories}
    \label{fig:graph}
\end{figure}

\Cref{fig:graph} represents the overall architecture of dependencies of the various theory files in the project,
with files further down the tree depending on those above.
\mintinline{isabelle}{[Pure]} and \mintinline{isabelle}{[HOL]} contain the base definitions necessary to do work in Isabelle,
all the rest are new theories created for the project.

From the simplest core definitions given in \mintinline{isabelle}{CA_Base},
the project splits into two due to the differences in implementing one dimensional versus two dimensional CA.
However the internal structure of these two subtrees is exactly the same apart from the distinction in dimension.
They both contain a base file for the two kinds of finite CA,
and another file dealing with the infinite case.

Each of the six root nodes contains the definition of one of the six distinct kinds of CA realised in the project.
They very roughly increase in power and complexity from left to right.

\section{The Cellular Automata Type}

The type of \mintinline{isabelle}{Cellular Automata} is redefined for each of the four main branches.
Those are elementary finite,
two-dimensional finite,
elementary infinite,
and two-dimensional infinite.
This is due to the geometric and functional differences that have to exist in the \mintinline{isabelle}{state} type parameter underlying each broad class of CA.
Despite this, the actual signature of the type remains unchanged for each.
This is to allow the definitions and proofs that sit on top of them to make the same assumptions about type structure and composition.
As shown below in \Cref{code:ca_def} the definition is very compact and simple.
All CA also share the binary cell type given in \Cref{code:cell_def}.

As one might expect,
a Cellular Automata consists only of the current global state it is in,
and the rule necessary to transition it to the next state.
This however does hide quite a lot of complexity,
and in fact the actual functioning of a CA is created from more than these two parameters.

As an example of this,
\cite{yu} actually defines CA as a 4-tuple $(k, S, N, f)$,
where $k$ is the dimension,
$S$ the set of states,
$N$ the neighbourhood,
and $f$ the local rule.
This is certainly more transparent then the definition given here.
In this project's version the information about neighbourhoods is given implicitly from a neighbourhood function that sits in the local context where the CA is actually run.
The reason for the approach taken here is for elegance and ease of use in further results on the CA type.
Adding more information to the type directly can mean more work in pattern matching,
and constructing and deconstructing the type every time it is used.

\begin{myminted}{Cell definition}{cell_def}
    datatype cell = Zero | One
\end{myminted}

\begin{myminted}{CA type signature}{ca_def}
    datatype CA = CA (State : state) (Rule : rule)
\end{myminted}

The \mintinline{isabelle}{State} and \mintinline{isabelle}{Rule} that are capitalised are just syntactic sugar for defining accessor functions for those arguments to the type.
So the \mintinline{isabelle}{State} function takes a CA and returns its state, 
and similar for \mintinline{isabelle}{Rule}.

It is also worth noting that the CA here is not directly defined as a tuple.
Using a unique datatype carries more semantic weight then giving a \mintinline{isabelle}{type_synonym} to a tuple.
The same concept does not exist in general mathematics so purely mathematical approaches to formalisms involve a tuple.

\subsection{State}

As mentioned in the above paragraphs,
the \mintinline{isabelle}{state} type is designed differently across the four general distinctions of CA.
As such the more detailed description of each is left to its own respective section.
For a high level overview it suffices to say that the finite CA states were simply a one or two dimensional list of states,
given geometry through either pattern matching or indices.
The infinite CA states were modeled completely differently,
as a function from the integers to cells.

\subsection{Neighbourhoods \& Rules}

Unlike \mintinline{isabelle}{state},
\mintinline{isabelle}{rule} is defined exactly the same way for all CA, as given in \Cref{code:rule_def}.
The only differing factor being the type of neighbourhood it acts on.

\begin{myminted}{Type definition of \mintinline{isabelle}{rule}}{rule_def}
    type_synonym rule = "neighbourhood => cell"
\end{myminted}

Neighbourhoods differ between one and two dimensions as expected but not in a conceptually deep way.
As shown in \Cref{code:nb_def} the only practical difference is adding more \mintinline{isabelle}{cell} arguments.

\begin{myminted}{The two kinds of neighbourhood}{nb_def}
    (* One dimensional *)
    datatype neighbourhood = Nb cell cell cell

    (* Two dimensional *)
    datatype neighbourhood = Nb
    (NorthWest:cell) (North:cell)  (NorthEast:cell)
    (West:cell)      (Centre:cell) (East:cell)
    (SouthWest:cell) (South:cell)  (SouthEast:cell)
\end{myminted}

This approach of generating neighbourhoods as an entity distinct from cells allowed a higher level approach to dealing with CA.
It meant that in the finite elementary CA,
the question of updating cells is solved by simply mapping a \mintinline{isabelle}{rule} over the neighbourhoods,
see \Cref{code:update_nbhd}.
By abstracting out a neighbourhood function it also allows for minor tweaks that entirely change the topology of a finite CA,
as explained in \Cref{sec:finite_CA}.


\begin{myminted}{Updating via neighbourhoods}{update_nbhd}
    (* Finite one dimensional *)
    fun update_CA :: "CA => CA" where
    "update_CA (CA s r) = CA (map r (nbhds s)) r"

    (* Finite two dimensional *)
    fun update_CA :: "CA => CA" where
    "update_CA (CA s r) = CA (map (λ xs. map r xs) (nbhds s)) r"
\end{myminted}

In the two dimensional finite case,
as the state of a CA is just a list of lists of cells,
it requires a mapping of maps over those but the general principle is similar.

A CA can be run for multiple steps through repeated application of \inline{update_CA} through another function designed for this.


\section{Finite Cellular Automata} \label{sec:finite_CA}

There were two separate paths taken in dealing with cells on the boundary of the finite CA.
The first,
which for the purposes of the project is named \emph{bounded},
is the simplest conceptually,
but not necessarily in implementation.
The other is termed \emph{toroidal},
because using it gives the CA the geometry of a torus or band.
Despite seeming more complicated,
implementation is very natural compared to the bounded method.

Bounded CA deal with the problem of determining the neighbourhood of a cell on the edge simply via ``padding''.
What this practically means is that when the neighbourhood function goes to generate the neighbourhood of a cell it knows to be on the edge,
it pretends that there are actually \mintinline{isabelle}{Zero} cells filling in the blanks and returns a neighbourhood accordingly.

The toroidal method is so called as it essentially ``glues'' the opposite ends of the \mintinline{isabelle}{state} data structure together,
thus creating a torus in two dimensions,
or a band in one.
From a purely mathematical topology perspective this works via altering the neighbourhoods of cells on the extremities,
to include those cells that should be next to it if it were in a toroidal shape.


\subsection{One Dimensional} 

In the one dimensional case this is very easy to implement.
All that has to be done is stick the additional cells onto the state list,
and call another function \mintinline{isabelle}{inner_nbhds} which performs the obvious action of stepping through the internal items in a list,
returning the neighbourhoods for each.

\begin{myminted}{Finite one dimensional neighbourhood generation}{finite_1D_nb}
    type_synonym state = "cell list"

    (* Bounded elementary CA *)
    fun nbhds :: "state => neighbourhood list" where
    "nbhds  xs = inner_nbhds (Zero # xs @ [Zero])"

    (* Toroidal elementary CA *)
    fun nbhds :: "state => neighbourhood list" where
    "nbhds xs = inner_nbhds ((last xs) # xs @ [hd xs])"
\end{myminted}

For the bounded automata these additional cells are defined to always be constant \mintinline{isabelle}{Zero} cells.
It would be equally valid and simple to take these both to be \mintinline{isabelle}{One} or a mix of each,
but zeroing the boundaries was the most natural choice.

As shown in \Cref{code:finite_1D_nb},
it only takes a very minor adjustment to the \mintinline{isabelle}{nbhd} function to radically overhaul the topology of the shape produced.
By changing the cells padded onto the state list to be the last item and head of the original list,
the cells of each extreme end of the list are associated with each other.

% TODO mention this as a possible n-dim extension later
Theoretically altering the neighbourhood function is all that is needed to turn a one dimensional list into any finite shape or geometry required.
However this would require much greater modifications to the base structure of the function
and so was not the approach taken in transitioning finite CA to two dimensions.


\subsection{Two dimensional}

If a one dimensional CA state uses a single list,
the obvious transition to scale to two dimensions is to use a two dimensional list.
% TODO mention drawbacks in Chap5

\begin{myminted}{Finite two dimensional neighbourhood generation}{finite_2D_nb}
    type_synonym state = "cell list list"

    (* function for all 2D neighbourhoods *)
    fun nbhds :: "state => neighbourhood list list" where
        "nbhds s = (let h = (int_height s)-1
        in (let w = (int_width s)-1 in
        [[get_nbhd s x y. y ← [0..h]]. x ← [0..w]]))"

    (* Toroidal get_nbhd *)
    fun get_nbhd :: "state => int => int => neighbourhood" where
        "get_nbhd s x y =
        (let w = int_width s in
        (let h = int_height s in
        list_to_nb [get_cell s ((x+i) mod w) ((y+j) mod h).
            j ← rev [-1..1], i ← [-1..1]]))"

    (* Bounded get_cell *)
    fun get_cell :: "state => int => int => cell" where
        "get_cell s x y = (if out_of_bounds s x y
                           then Zero
                           else  s!(nat x)!(nat y))"
\end{myminted}

It is very clear from \Cref{code:finite_2D_nb} that the move to two dimensions involved the use of a lot more indices in lists.
This is due to the fact that pattern matching is not as simple when you have lists of lists,
so indexing the lists was deemed easier to work with.

The main \inline{nbhds} function achieves this by a nested list comprehension over the length and width of the CA state,
and fills in those indices with \inline{get_nbhd}.
The function \inline{get_nbhd} itself delegates to \inline{get_cell}.

In this case both the bounded and toroidal CA have the same \mintinline{isabelle}{nbhds} function,
the different shapes are instead produced through changes to the intermediary \mintinline{isabelle}{get_nbhd}, 
and \mintinline{isabelle}{get_cell}.
Note how the \mintinline{isabelle}{get_nbhd} function does the work to create a torus by wrapping indices around with \mintinline{isabelle}{mod},
while in the bounded case \mintinline{isabelle}{get_cell} is responsible for filling in the \mintinline{isabelle}{Zero}s.

For the toroidal CA, \inline{get_cell} does not default to returning \inline{Zero},
and for bounded CA,
\inline{get_nbhd} does not use any modular wrapping.

To understand the indexing used in the two dimensional state,
the inner lists have to be thought of as column vectors moving vertically. % TODO phrase better?


\section{Infinite Cellular Automata}

Infinite CA are simpler because they have no boundaries meaning there was no need for the all the edge case handling that characterised finite CA.
However they require discarding conventional data structures like lists,
for an approach much more grounded in mathematics and functional programming.


\subsection{Neighbourhoods}

The neighbourhood does not play quite as big a part in the infinite CA,
especially since there were no variations made of the topology.
However it was still used in the interest of consistency in level of abstraction and definition,
and to allow for potential future uses of it.
This allowed the type signature of \inline{rule} to remain unchanged too.


\subsection{State}

State in infinite CA,
and extending it,
turned out to be much simpler than the finite case.
One dimensional state simply needs to pair every integer with a cell,
which naturally works as a function over the integers.
Note this function is total so the entire state is represented,
there is no need for a partial function or something of that kind.

Two dimensional state is extended easily from the base case,
although strictly speaking it is not modelled as a function from the Cartesian product of the integers,
but just as a function with two integer arguments.
These two are isomorphic,
but multiple arguments are more easily extensible,
and allows for partial application and currying.

\begin{myminted}{Infinite state represented as functions}{state_infinite}
    (* One dimensional *)
    type_synonym state = "int => cell"

    (* Two dimensional *)
    type_synonym state = "int => int => cell"
\end{myminted}

Unlike the finite case,
the real work for infinite CA is done in updating the state.
Using \inline{map} was no longer an option as there is no finite data structure to map over.
Instead a recursive tactic was employed.
The whole state never explicitly exists at once,
it cannot as it is infinite,
but the values of any amount of cells at any stage can always be known.
The key insight that made this work,
was to update the \inline{state} function at each stage,
with the new action on an integer recursively based on what the cell values around it were in the previous stage.
So when the function is called after a certain amount of updates,
it calculates the current values needed by recursing all the way back to a base state that was supplied.
This works very nicely in one dimension,
but the two dimensional version can no longer be considered elegant.

\begin{myminted}{Infinite transition functions}{update_infinite}
    (* One dimensional *)
    fun update_state :: "CA => state" where
    "update_state (CA s r) n = r (Nb (s (n-1)) (s n) (s (n+1)))"

    (* Two dimensional *)
    fun update_state :: "CA => state" where
    "update_state (CA s r) x y =
        r (Nb (s (x-1) (y+1)) (s x (y+1)) (s (x+1) (y+1))
              (s (x-1)  y)    (s x  y)    (s (x+1)  y)
              (s (x-1) (y-1)) (s x (y-1)) (s (x+1) (y-1)))"
\end{myminted}

Overall due to the definitions used,
the differences in implementation between one and two dimensional infinite automata,
were much smaller than the differences in the finite case.


\section{Properties}

Due to the way CA and their constituent components were built with the same type signatures,
most properties built on them could hold with mostly trivial adjustments for the CA type.
Additionally a lot of common properties about CA, 
when phrased in the language to describe CA developed here,
are actually properties about the rule making up a CA.
These abstractions enabled some properties to apply with no adjustments to all CA types.

In total about sixteen distinct properties of CA were defined,
some totally generic,
others,
like those in \Cref{code:props_1D},
are specific to the structural constraints of one dimension.
This number also does not include utility functions necessary to get the CA to work,
along with ancillary lemmas proved about these for basic simplification properties.
Additionally for each of these properties,
they were related to either another property or a concrete CA via theorems and lemmas.

\begin{myminted}{One dimensional structure properties}{props_1D}
    fun mirror :: "rule => rule" where
    "mirror r (Nb a b c) = r (Nb c b a)"

    definition amphichiral :: "rule => bool" where
    "amphichiral r ≡ (ALL c. r c = (mirror r) c)"
\end{myminted}

Note the \inline{ALL} is simply an ASCII stand in for the $\forall$ quantifier.


As well as properties about CA types in general,
two specific rules of interest were defined with some properties proven about them.
These were Rule 110 for one dimensional CA and Conway's Game of Life for two dimensions.

\subsection{Rule 110}
Rule 110 is an interesting example as it has been proven Turing Complete.
Some negative results were proved about it,
for instance it can be shown that it does not fit the definition of amphichiral from \Cref{code:props_1D}.

\begin{myminted}{Negative result on Rule 110}{r110_neg}
    lemma "¬amphichiral r110"
by (metis amphichiral_def cell.distinct(1) mirror.simps r110.simps(4) r110.simps(7))
\end{myminted}

This lemma does not have as nice as proof as that in \Cref{code:life_total}.
This is because it is a computer generated proof,
found via use of \inline{sledgehammer}.
Many proofs proofs in this project were found via computer search,
a hallmark of this is a proof being very short but not very human readable.

\subsection{Game of Life}
The Game of Life is a two dimensional CA popularised by John Conway and Martin Gardener.
Like Rule 110 it is also Turing Complete.
One key aspect of Life,
is that is it \emph{outer totalistic},
meaning a cell changes depending only on its own state
and the total number of \inline{One} cells in its neighbourhood.
Translated into Isabelle we get the following \Cref{code:life_total},
which contains a one line proof of this simply via ``unfurling'' the definitions.

\begin{myminted}{Outer totalistic Life}{life_total}
    definition outer_totalistic :: "rule => bool" where
    "outer_totalistic r ≡ (ALL nb1 nb2. (Centre nb1 = Centre nb2)
    --> (outer_sum nb1 = outer_sum nb2) --> (r nb1) = (r nb2))"

    theorem "outer_totalistic life"
    by apply(simp add: outer_totalistic_def life_def)
\end{myminted}
