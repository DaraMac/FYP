\chapter{The Problem}

\section{Problem Analysis}

The goal of the project was to somehow implement a variety of types of CA in Isabelle.
The types of CA were one or two dimensional CA that were either finite or infinite,
leading to essentially four main categories to develop.
There were two main issues to address for each of these.

The first was how to implicitly or explicitly represent the state of a CA.
This question was especially key in the infinite case, 
as it requires thinking beyond traditional finite data structures.
The other topic of interest was how to develop the state of the CA as time progresses.
Again this is much more a difficult and important issue in the infinite case,
as you cannot simply apply a function over infinitely many values.
This rules out the obvious choices such as using \mintinline{isabelle}{map} and other functions in a similar vein.

The finite automata also had certain issues that do not occur in the infinite cases.
As they are finite they have boundaries,
and for cells on these boundaries the neighbourhood necessary to determine its evolution does not exist in the standard way.
This requires a decision to be made to either adjust the neighbourhood of these cells,
or the rule that applies to them.
