\chapter{Evaluation}

\section{Correctness on a Software Level}

As mentioned in \Cref{sec:metrics} trying to quantitatively evaluate a mathematical statement or definition is not an easy task.
The most crude method but also the most essential is to check whether the results from it match up with the expected results.
As there is not another convenient, computerised, and provably correct formalisation of Cellular Automata,
some of this must be done by hand.
To this end all the CA types were instantiated on a variety of simple example programs, 
and the results after different numbers of iterations checked against robust pre-existing simulators.
Although there were some issues, 
all current CA passed these tests.
As CA are very simple models,
testing a certain number of base and edge cases gives enough information for extrapolation.


\section{Analysis of Models}
However all this shows is that these programs achieve the goal of modeling CA.
These tests do not answer the much more important and nuanced question of whether they do this well.

One mark in favour of this is that it was possible to quite succinctly define properties the CA models should have,
and to relatively easily prove that certain example CA like Rule 110 did or did not have these properties.

One issue was code duplication.
In the absence of any grand unifying model or typeclass for all CA,
properties often had to be restated with no changes in multiple different files.
This is an inefficient method that would not be very scalable if this were a longer term project.

The CA models themselves,
with the exception of the two dimensional finite ones,
are short and for the most part intuitively clear in their aims.
The type keeps things to the bare essentials,
and introduction of structures like neighbourhoods means the whole action of a CA is produced from the composition of a few, very simple functions.
This is certainly a success in terms of what was hoped for at the start of the project.

In this regard the best models are the elementary infinite CA
and both the finite elementary CA due to the simplicity of their implementations.
Unlike those, the two dimensional infinite CA has a lot of baggage when it is required to manually deal with all nine cells in a neighbourhood.
The definitions of two dimensional finite CA certainly match up the least with the informal description.
From reading their implementations their purpose is not as immediately obvious as with the others.
