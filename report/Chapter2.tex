\chapter{Technical Background}

\section{Topic Material}

\subsection{Formalisations in Computability Theory}

As mentioned before,
this is certainly not the first formalisation of computability theory concepts inside of proof assistants.
Other works however have focused on more traditional models or with certain specific proofs in mind.
These include $\lambda$-calculus in Coq \cite{forster},
partial recursive functions in Lean \cite{carneiro},
and the aforementioned Turing and Abacus machines in Isabelle \cite{urban}.
Other than choices of model and language,
the key difference between those works and this project is the end goal.
The three mentioned above all set about to formalise computability theory from the perspective of their chosen model or models.
This project works with just Cellular Automata and what results can be specifically shown about them.

Additionally the approach in \cite{urban} often involves specifying certain exact indices in lists and having to manipulate these numbers.
This obviously has its technical benefits and is necessary when working with TMs,
but the desire in this work was to take a ``wholemeal'' functional programming approach to the construction where possible.

The existing research that comes closest to both the goals and execution of this project,
is a formalisation of CA in Coq for the purposes of verifying a result about the firing squad problem \cite{firing}.
Their work bases the CA over the naturals $\mathbb{N}$,
and includes an explicit time parameter.
As well as that,
their transition function is based more around individual cells,
and the properties they define are all based around eventually deriving the proof.
This project certainly has similarities,
but trades the depth of \cite{firing} for breadth,
and makes use of definitions that try to be more compact.
For instance the time parameter is more external to the cellular automata, and the varieties of different CA are defined with different mechanisms.


\subsection{Cellular Automata}
The concept of CA has been around since the 1940s and were popularised in the 70s with Conway's Game of Life.
However a lot of the names and terminology associated with them comes from Stephen Wolfram \cite{wolfram}.
His work is not entirely formal,
but others have provided their own formalisations of definitions he put forward \cite{yu}.
This was especially useful as an aid in translating the some of the classifications of CA into Isabelle, 
even if their formalisation differed in many other ways.


\section{Technical Material}

\subsection{Cellular Automata}
